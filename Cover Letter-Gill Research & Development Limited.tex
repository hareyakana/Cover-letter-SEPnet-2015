\documentclass[11pt]{letter} % Default font size of the document, change to 10pt to fit more text

\usepackage{newcent} % Default font is the New Century Schoolbook PostScript font 
\usepackage{helvet} % Uncomment this (while commenting the above line) to use the Helvetica font

% Margins
\topmargin=-1in % Moves the top of the document 1 inch above the default
\textheight=10in % Total height of the text on the page before text goes on to the next page, this can be increased in a longer letter
\oddsidemargin=-10pt % Position of the left margin, can be negative or positive if you want more or less room
\textwidth=6.5in % Total width of the text, increase this if the left margin was decreased and vice-versa

\let\raggedleft\raggedright % Pushes the date (at the top) to the left, comment this line to have the date on the right

\begin{document}

%----------------------------------------------------------------------------------------
%	ADDRESSEE SECTION
%----------------------------------------------------------------------------------------

\begin{letter}{Ollie Blacklock\\oliver.blacklock@gillrd.com\\Tel:01590 613405/01590 613400\\Gill R\&DLtd.,\\Saltmarsh Park,\\67 Gosport Street,\\Lymington,\\Hants,\\ SO41 9EG}

%----------------------------------------------------------------------------------------
%	LETTER CONTENT SECTION
%----------------------------------------------------------------------------------------

\opening{Ollie Blacklock,} 

I am writing this letter to express my interest in SEPnet Summer Placement Project "New fluid flow system prototype development". I am a third year MPhys student at the University of Southampton. 

My personal goal is to an experimental particle physicist therefore I am looking for opportunity to gain experimental/practical experience as a physicist. This placement project would be of my interest to pursue as I has also liked doing practical experimental work. I have learnt some programming in python through some courses in my university and some shell command from a computing bootcamp last year. I would not say I am capable to do coding  but I believe with proper tools from the internet I can get things done for the experimental purposes.

In my last summer, I did a 8 week placement with Oxford Instruments working on a shrink fit problem of a NbTi magnet . In the project, I had to design the experimental pieces and carried out the experiment by myself using the tools provides with advices from my supervisor. In that project, I have used software like labview to communicate equipment with the computer and analyse it on the computer. Handling of 17 tonne press and liquid nitrogen were heavily involved throughout the project. The result of the project was later used in other magnets that required similar fitting technique. Therefore I am pretty confident that experimental task would be pretty smooth for me to get on and with area that I could improve from previous placement is that writing a more thorough technical report with more consideration on errors. I am not good enough by industrial standard but experimental work would be something i enjoy doing.

Though engineering is something I would consider as a career option, I am leaning more towards doing a phd though it depend on finical situation after my degree. I do however feel that the hand-ons expect of developing new technology with combination quantum mechanical aspect of physics is something I consider worth doing as I would like keep my options as wide as possible. I am willing to gladly accept any interviews if needed and travel for it within my abilities.

%dissertation -plasmon/neutrino oscillation,physics beyond the standard model
%oxford sepnet placement,coding experience,


\closing{Sincerely,\\Ken Keong Lee}


%\encl{Curriculum vitae, employment form} % List your enclosed documents here, comment this out to get rid of the "encl:"

%----------------------------------------------------------------------------------------

\end{letter}

\end{document}