\documentclass[11pt]{letter} % Default font size of the document, change to 10pt to fit more text

\usepackage{newcent} % Default font is the New Century Schoolbook PostScript font 
\usepackage{helvet} % Uncomment this (while commenting the above line) to use the Helvetica font

% Margins
\topmargin=-1in % Moves the top of the document 1 inch above the default
\textheight=10in % Total height of the text on the page before text goes on to the next page, this can be increased in a longer letter
\oddsidemargin=-10pt % Position of the left margin, can be negative or positive if you want more or less room
\textwidth=6.5in % Total width of the text, increase this if the left margin was decreased and vice-versa

\let\raggedleft\raggedright % Pushes the date (at the top) to the left, comment this line to have the date on the right

\begin{document}

%----------------------------------------------------------------------------------------
%	ADDRESSEE SECTION
%----------------------------------------------------------------------------------------

\begin{letter}{Ian Giles\\ian.giles@phoenixphotonics.com\\Tel:01843 0843709\\Phoenix Photonics Ltd,\\Sarre Business Centre,\\Canterbury Road,\\Sarre CT7 0JZ}

%----------------------------------------------------------------------------------------
%	LETTER CONTENT SECTION
%----------------------------------------------------------------------------------------

\opening{Ian Giles,} 

I am writing this letter to express my interest in SEPnet Summer Placement Project "Fibre optic components and instruments". I am a third year MPhys student at the University of Southampton.

My goal is to an experimental particle physicist, and therefore I am looking opportunities to improve my practical skills as a physicist therefore experimental based project are more appealing option for me which this project interest me. I have took every lab module possible in my university so far as i enjoyed learning physics through experiment rather than lecture and reading through literature review of some hard to teach topic. There a few aspect that I wish to improve on such as better understand in the technical report and data/error analysis through programming since each opportunity I get to work on are precious experience for me as it is not frequent that I can work on an experiment for extended period of time.

I took the 2 photonics module in my course so I would say that I have fair understanding of optical physics and handling of optical tools based experiment are much frequent in the lab modules I took in my course. For my photonics essay, I have written on the topic of plasmonics and I would not be too foreign on the topics relating to fibre optics based on what I have learnt about it so far. Since my previous placement was with a reasonably big company, I would love to experience working in a smaller group of people.

In my last summer, I did a placement with Oxford Instruments working on a shrink fit problem of a NbTi magnet. I was task from designing the experimental set-up to analysis the result of the experiment. The project involves heavily on handling of a 17 tons press and liquid nitrogen with minimal supervision. The result of the project was later then used in their other products which required a similar fitting technique. I enjoyed the work I did there, so this project about fibre optics components would be something that I would enjoy.

I would be gladly accept and travel for the interview if needed if it is within my ability to do so.

%dissertation -plasmon/neutrino oscillation,physics beyond the standard model
%oxford sepnet placement,coding experience,


\closing{Sincerely,\\Ken Keong Lee}


%\encl{Curriculum vitae, employment form} % List your enclosed documents here, comment this out to get rid of the "encl:"

%----------------------------------------------------------------------------------------

\end{letter}

\end{document}