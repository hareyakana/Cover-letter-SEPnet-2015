\documentclass[11pt]{letter} % Default font size of the document, change to 10pt to fit more text

\usepackage{newcent} % Default font is the New Century Schoolbook PostScript font 
\usepackage{helvet} % Uncomment this (while commenting the above line) to use the Helvetica font

% Margins
\topmargin=-1in % Moves the top of the document 1 inch above the default
\textheight=8.5in % Total height of the text on the page before text goes on to the next page, this can be increased in a longer letter
\oddsidemargin=-10pt % Position of the left margin, can be negative or positive if you want more or less room
\textwidth=6.5in % Total width of the text, increase this if the left margin was decreased and vice-versa

\let\raggedleft\raggedright % Pushes the date (at the top) to the left, comment this line to have the date on the right

\begin{document}

%----------------------------------------------------------------------------------------
%	ADDRESSEE SECTION
%----------------------------------------------------------------------------------------

\begin{letter}{Professor Alexander Belyaev\\alexander.belyaev@cern.ch\\Tel:02380 598509\\Rutherford Appleton Laboratory,\\CMS group,\\Research Division B,\\Particle Physics Department,\\Didcot OX11 0QX}

%----------------------------------------------------------------------------------------
%	LETTER CONTENT SECTION
%----------------------------------------------------------------------------------------

\opening{Professor Alexander Belyaev,} 

I am writing this letter to express my interest in SEPnet Summer Placement Project "Exploring Higgs boson Physics beyond the Standard Model with High Energy Physics Model Data Base(HEPMDB)". I found that this project would enrich my skills towards being an experimental physicist. I 
%-dissertation -plasmon/neutrino oscillation,physics beyond the standard model
%-oxford sepnet placement,coding experience,


\closing{Sincerely,\\Ken Keong Lee}


%\encl{Curriculum vitae, employment form} % List your enclosed documents here, comment this out to get rid of the "encl:"

%----------------------------------------------------------------------------------------

\end{letter}

\end{document}