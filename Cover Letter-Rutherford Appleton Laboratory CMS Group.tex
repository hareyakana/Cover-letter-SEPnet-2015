\documentclass[10pt]{letter} % Default font size of the document, change to 10pt to fit more text

\usepackage{newcent} % Default font is the New Century Schoolbook PostScript font 
\usepackage{helvet} % Uncomment this (while commenting the above line) to use the Helvetica font

% Margins
\topmargin=-1in % Moves the top of the document 1 inch above the default
\textheight=8.5in % Total height of the text on the page before text goes on to the next page, this can be increased in a longer letter
\oddsidemargin=-10pt % Position of the left margin, can be negative or positive if you want more or less room
\textwidth=6.5in % Total width of the text, increase this if the left margin was decreased and vice-versa

%\let\raggedleft\raggedright % Pushes the date (at the top) to the left, comment this line to have the date on the right

\begin{document}

%----------------------------------------------------------------------------------------
%	ADDRESSEE SECTION
%----------------------------------------------------------------------------------------

\begin{letter}{Professor Alexander Belyaev\\alexander.belyaev@cern.ch\\Tel:02380 598509\\Rutherford Appleton Laboratory,\\CMS group,\\Research Division B,\\Particle Physics Department,\\Didcot OX11 0QX}

%----------------------------------------------------------------------------------------
%	LETTER CONTENT SECTION
%----------------------------------------------------------------------------------------

\opening{Professor Alexander Belyaev,} 

I am writing this letter to express my interest in SEPnet Summer Placement Project "Exploring Higgs boson Physics beyond the Standard Model with High Energy Physics Model Data Base(HEPMDB)". I am a third year MPhys student at the University of Southampton. I have plans of doing a possible phd after my undergraduate studies but still have some doubts on it due to personal reasons. Therefore through placement like this I believe would help me solidify some of the decision I have for after my degree.

I feel that this placement would enhanced my skills towards my goal of being an experimental particle physicist. Another reason for applying is that through this placement that my understanding of the Standard Model and the physics beyond it would improve greatly as it is my interest within this field of physics. Though my computer programming skills is lacking in certain area although I have some exposure in using python/labview/unix-shell command, I believe with more usage of different language would improve my programming skill needed as a physicist.

For some of the dissertation I did during the recent 2 years of my university life, I have studied on the topics of plasmonics for my photonics modules in my 2nd year and physics beyond the Standard Model specifically on the topics of neutrino oscillation for my dissertation module in my 3rd year. I taken modules on particle physics and it help me to understood the Standard Model better but i still feel that i have much more to learn as a physicist. In my dissertation on neutrino oscillation, I have learnt that the 3 flavour of neutrino in the Standard Model is not massless as experimental result in the last 2 decades have proven so. Therefore i believe through placement such as this would stir my passion for particle physics even further as I endeavour more in this field of Physics.

It is also worth mentioning that i have done a summer placement last summer with Oxford instruments for 8 weeks and i am quite comfortable in working independently and also in groups. I have worked on studying the shrink fit of a NbTi magnet and the result was used in their products that required this technique in manufacturing. I would be gladly to accept any interview and travel for it if it is within my capability to do so.
%-dissertation -plasmon/neutrino oscillation,physics beyond the standard model
%-oxford sepnet placement,coding experience,


\closing{Sincerely,\\Ken Keong Lee}


%\encl{Curriculum vitae, employment form} % List your enclosed documents here, comment this out to get rid of the "encl:"

%----------------------------------------------------------------------------------------

\end{letter}

\end{document}