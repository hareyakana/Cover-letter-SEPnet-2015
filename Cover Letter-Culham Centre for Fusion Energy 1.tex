\documentclass[11pt]{letter} % Default font size of the document, change to 10pt to fit more text

\usepackage{newcent} % Default font is the New Century Schoolbook PostScript font 
\usepackage{helvet} % Uncomment this (while commenting the above line) to use the Helvetica font

% Margins
\topmargin=-1in % Moves the top of the document 1 inch above the default
\textheight=10in % Total height of the text on the page before text goes on to the next page, this can be increased in a longer letter
\oddsidemargin=-10pt % Position of the left margin, can be negative or positive if you want more or less room
\textwidth=6.5in % Total width of the text, increase this if the left margin was decreased and vice-versa

\let\raggedleft\raggedright % Pushes the date (at the top) to the left, comment this line to have the date on the right

\begin{document}

%----------------------------------------------------------------------------------------
%	ADDRESSEE SECTION
%----------------------------------------------------------------------------------------

\begin{letter}{Rachel Lawless\\Rachel.Lawless@ccfe.ac.uk\\Tel:01235 464656\\Fulham Centre for Fusion Energy,\\Culham Science Centre,\\Abingdon,\\Oxfordshire OX14 3EA}

%----------------------------------------------------------------------------------------
%	LETTER CONTENT SECTION
%----------------------------------------------------------------------------------------

\opening{Rachel Lawless,} 

I am writing this letter to express my interest in SEPnet Summer Placement Project "Tritium detection and accountancy research in support of conceptual designs of fusion reactors and operating fusion facilities". I am a third year MPhys student at the University of Southampton.

For my dissertation, I have done a short review on the topics of neutrino oscillation connection to the physics beyond the Standard Model. I have come across double beta tritium decay which are currently among the research topics in the field of neutrino physics. Therefore this project of tritium detection particularly interest me as it would great improve my understanding of physics in this area. I am more of an experimentalist rather than theorist and one day hoping to become an experimental particle physicist. Therefore experimental task involving data analysis and designing experiment is always I am happy to do.

In my last summer, I have done a 8 week placement with Oxford Instruments where i am task with studying the shrink fit or interference fit of a NbTi magnet. The result of the placement were later applied to their other products which required similar technique. I was task from designing the experiment concept to analyse the data and it was a pleasant experiment for myself as I gain a confidence boost of doing similar task. Therefore understanding the physics of a system and write a review on it would be something that I am fairly confident would be able to do. 

Learning new physics concept is also something that I would enjoy and mixed discipline topics with physics is also something I would consider along my career choices. I understand that coding/programming is something that is getting more and more important in every career choice and I would love to improve myself on this aspect through using different software and tools to complete certain task. Though I have experience in using pythons, labview and etc., I aware that I have still much more to learn and hopefully be fairly confident in this kind of task.

I am gladly to accept any interview if needed within my own capability though I do not have a car but willing to travel.
%an opportunity to help my decision to do a phd or getting a job
%enrich personal skills as an experimentalist
%-dissertation -plasmon/neutrino oscillation,physics beyond the standard model
%-oxford sepnet placement,coding experience,


\closing{Sincerely,\\Ken Keong Lee}


%\encl{Curriculum vitae, employment form} % List your enclosed documents here, comment this out to get rid of the "encl:"

%----------------------------------------------------------------------------------------

\end{letter}

\end{document}