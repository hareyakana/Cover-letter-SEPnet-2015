\documentclass[11pt]{letter} % Default font size of the document, change to 10pt to fit more text

\usepackage{newcent} % Default font is the New Century Schoolbook PostScript font 
\usepackage{helvet} % Uncomment this (while commenting the above line) to use the Helvetica font

% Margins
\topmargin=-1in % Moves the top of the document 1 inch above the default
\textheight=10in % Total height of the text on the page before text goes on to the next page, this can be increased in a longer letter
\oddsidemargin=-10pt % Position of the left margin, can be negative or positive if you want more or less room
\textwidth=6.5in % Total width of the text, increase this if the left margin was decreased and vice-versa

\let\raggedleft\raggedright % Pushes the date (at the top) to the left, comment this line to have the date on the right

\begin{document}

%----------------------------------------------------------------------------------------
%	ADDRESSEE SECTION
%----------------------------------------------------------------------------------------

\begin{letter}{Rob Felton\\Robert.Felton@ccfe.ac.uk\\tel:01235 46533\\Culham Centre for Fusion Energy,\\Culham Science Centre,\\Abingdon,\\Oxfordshire OX14 3EA}

%----------------------------------------------------------------------------------------
%	LETTER CONTENT SECTION
%----------------------------------------------------------------------------------------

\opening{Rob Felton,} 

I am writing this letter to express my interest in SEPnet Summer Placement Project "Experimental Physics Models". I am a third year MPhys student at the University of Southampton. This placement project interest me especially in the aspect of the  experimental approach in studying plasma models and also an opportunity to improve my practical skills as a physicist in the area of data analysis and computer modelling.

In my last summer, I did an 8 week placement with Oxford Instruments working on a shrink fit problem of a NbTi magnet which in many aspect of it I find there is many similarity of a more engineering approach rather than complex physics. The result from it was later used in their other products which required the same technique of constructing the magnet. With my experience from my last summer placement, I feel that I could improve in some aspect such as a more thorough technical report which I could contribute to this placement project. Finite element analysis (ANSYS) is something I learnt in the last summer which I feel in many ways would better help me understand the plasma model that this placement has to offer. Though in the area of programming I am lacking the confidence, I have some experience using python and labview which may be useful in analysing the data obtained from experiment.

In one of my photonics modules, part of the coursework involved writing an essay about something beyond the scope of the course and I choose plasmon as the main topic of my essay. Though it is totally different from what the plasma does in fusion energy, it would be quite refreshing for myself to learn something new and expand my current knowledge of physics. Therefore learning physics from literature review is something that I am fairly capable of doing so as demonstrate in the dissertation I did in my last semester. Also hearing about that the involvement of my last employer, Oxford Instrument have involved with the development of the magnets at Tokamak during my last placement, this project does feel familiar for myself.

I would gladly accept if there is any need for interview and would try my best to fulfil any request or information within my capabilities.

%-dissertation -plasmon/neutrino oscillation,physics beyond the standard model
%-oxford sepnet placement,coding experience,


\closing{Sincerely,\\Ken Keong Lee}


%\encl{Curriculum vitae, employment form} % List your enclosed documents here, comment this out to get rid of the "encl:"

%----------------------------------------------------------------------------------------

\end{letter}

\end{document}